\section{Zones}
\begin{flushleft}
	On d\'efinit par zones un espace en trois dimensions d'au moins 9 blocs de c\^ot\'e.\\
	Une zone peut \^etre cr\'ee par un joueur ou un administrateur. Une zone peut poss\'eder une ou plusieurs des caract\'eristiques suivantes :
	\begin{enumerate}
		\item un r\`eglement, du moment qu'il soit en accord avec le pr\'esent r\`eglement ou la Loi (voir introduction)
		\item une liste de joueurs autoris\'es a la fr\'equenter, l'exploiter, etc. .
	\end{enumerate}
	Une zone poss\`ede forc\'ement les caract\'eristiques suivantes :
	\begin{enumerate}
		\item Une limite minimum de taille
		\item Une limite maximum de taille
		\item Des fronti\`eres d\'efinies et approuv\'ees par :
		\begin{enumerate}
			\item Un administrateur
			\item Le maire de la ville
			\item L'administrateur a la pr\'ec\'edence.
		\end{enumerate}
	\end{enumerate}
	Une zone peut \^etre cr\'ee seulement si :
	\begin{enumerate}
		\item Elle ne chevauche pas une autre zone;
		\item Elle est d'une taille raisonnable;
		\item Le propri\'etaire de la zone n'a pas d\'ej\`a une zone adjactente.
		\item Elle se situe \àa \textit{au moins} 150 blocs de distance du spawn.
	\end{enumerate}
	La zone cr\'e\'ee peut ensuite \^etre utilis\'ee pour n'importe quel type d'action, \`a l'exception des actions dangereuses. Les zones dangereuses doivent \^etre constuitres de fa\c con \`a pouvoir contenir le risque \textbf{\`a l'int\'erieur de la zone}. \\
	La zone doit respecter le reste du r\`eglement une fois cr\'ee (voir sections suivantes).
\end{flushleft}
